\section{Theory}
Every laser consists of three main components, the gain medium, a pump source
and an optical resonator. 
Stimulated emission in the gain medium amplifies incoming light. For these atomic transitions,
which determine the laser spectrum, it is crucial to create population inversion in the 
gain medium. This is achieved by the pump source, which constantly supplies energy.
The amplification grows exponentially with the travel distrance of the photons through 
the gain medium. Therefore the gain medium is placed in a resonator consisting of two 
parallel reflecting elements so that the photons can pass the gain medium multiple times.

\subsection*{Atomic Transitions}
In atomic transitions between different discrete energy levels the three processes 
absorption, spontaneous emission and stimulated emission have to be distinguished, which are
shown schematically for a three-level system in Fig. \ref{fig:transitions}.
\begin{figure}
    \centering
    \includegraphics[width = 0.8\linewidth]{Bilder/transitions.png}
    \caption{Atomic transitions between two energy levels in a three-level system \cite{eichler}.}
    \label{fig:transitions}
\end{figure}
Spontaneous emission is the transition from a level of higher energy to a level of lower 
energy in which the energy difference between the two levels in released form of a photon.
If the transition is not spontaneous but induced by an incoming photon the process is 
called stimulated emission. 
Absorption occurs when an incoming photon causes a transition 
from the lower to the higher energy level and is absorbed in the process. 
In both stimulated emission and absorption the energy of the photon must correspond to the energy 
difference of the two energy levels.

Since the number of photons doubles in stimulated emission, light is amplified when 
stimulated emission occurs more often than spontaneous emission. For that, population inversion
is necessary and created by the pump source that adds energy to the 
system and thus increases the occupation probability of the upper level.                                                                                                                                                                                                                                                                                                                                                                                                                                                                                                                                                                                                         
Population inversion is not possible in a two-level system where only an equal occupation of 
both levels can be achieved because of the short relaxation time due to spontaneous emission.
Therefore at least three levels are needed. In this case, absorption occurs between the first (lowest)
and the third (highest) level, $E_1 \rightarrow E_3$. If the relaxation time of $E_3$ is 
much shorter than the relaxation time of $E_2$, a population inversion between the levels 
$E_1$ and $E_2$ is realised.

\subsection*{Diode Lasers}
In a diode laser a semiconductor is used as the medium and a electric current is used as a 
pump source. 
%The conductivity of a material depends on the occupation of the conducting band and the 
In semiconducting materials the band gap between the conducting band and the valence band is very small. %with typical energies of a few electronvolts 
%with the Fermi level lying between those bands. 
At temperatures close to $T = 0 \text{ K}$, the conducting band is empty and the valence 
band fully occupied so that the material is isolating. 
By heating, electrons can be excited from the vanlence band to the conducting band, leaving
a hole in the valence band, and the material becomes conductive. 
The conductive properties of a semiconductor can also be increased by doping the material,
that is, introducing impurities into the crystal structure. These impurities, called dopants, are atoms with a 
different number of valence electrons than the semiconducting material and 
can either be donors and acceptors. 
Donors have an additional valence electron that can propagate freely in the conducting band,
the material is then n-doted. 
Acceptors are missing one valence electron and create holes as a positive charge carrier, the 
material is p-doted.
%If donors are introduced into the material, it is n-doted, if acceptors are introduced, it is p-doted.
In both cases the energy bands are shifted relative to the Fermi level, resulting in an 
increased conductivity.

Connecting p- and n-doted areas creates a p-n-junction. 
\begin{figure}
    \centering
    \includegraphics[width = 0.6\linewidth]{Bilder/semiconductor.png}
    \caption{Schematic illustration of a diode laser as a p-n-junction. 
    Top: Energy bands of seperated semiconductors with n- (left) and p-doping (right). $F_C$ and $F_V$ are the Fermi-Levels
    corresponding to the conduction (CB) and valence band (VB).
    Middle: Space charge densitiy distribution resulting from a diffusion of 
    electrons into the p-region and a diffusion of holes into the n-region 
    when connecting n- and p-type semiconductors.
    Bottom: Electron energy in the connected n- and p-doped areas \cite{eichler}.}
    \label{fig:semiconductor}
\end{figure}
Light is emitted due to recombination of electron-hole pairs at the p-n-junction.
The wavelength of the resonator corresponds to the band gap of the material.
Two parallel layers of the medium, one of the only partially reflecting, function as the 
resonator, or internal cavity.