\section{Theory}
Every laser consists of three main components, the gain medium, a pump source
and an optical resonator. 
Stimulated emission in the gain medium amplifies incoming light. For these atomic transitions,
which determine the laser spectrum, it is crucial to create population inversion in the 
gain medium. This is achieved by the pump source, which constantly supplies energy.
The resonator


\subsection*{Atomic Transitions}
In atomic transitions between two discrete energy levels $E_1$ and $E_2$ the three different processes 
absorption, spontaneous emission and stimulated emission have to be distinguished as shown 
in Fig. %\ref{fig:transitions}.
Spontaneous emission is the transition from a state of higher energy to a state of lower 
energy in which the energy difference between the two states in released form of a photon.
If the transition is not spontaneous but induced by an incoming photon the process is 
called stimulated emission. 
Absorption occurs when an incoming photon is absorbed by the system and causes a transition 
from the lower to the higher energy state. 
In both stimulated emission and absorption the energy of the photon must correspond to the energy 
difference of the two energy states.
Light is amplified when stimulated emission occurs more often than spontaneous emission. 
The number of emitted photons doubles