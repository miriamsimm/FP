\section{Discussion}
Achieving the population inversion of the Laser, was managed through adjusting the vertical knob and the current of the Laser. The rapid Intensity gain was, as one expected and generally there where no problems with this step.
The second step, displaying the Laser inside the Rb gas, was also straightforward. 
The measurement of the Rb spectrum, was a little bit more difficult. Mode hopping mirrored the results, but could be mitigated by  adjusting the offset of the piezo-current. 
Another problem was the triangle background of the piezo-current. This could be mitigated by adjusting the gains of the photodiodes, which lead to a relatively straight line for the zero point of the spectrum \ref{fig:spectrum}.
Slight remnants of this can be still observed on the left edge of \ref{fig:sub1}, where the yellow line mirrors the blue triangle current. Between the measured spectrum \ref{fig:sub1} and the expected spectrum \ref{fig:sub2} one observes different relative heights for some of the peaks. 
Overall the general structure of the peaks is in good Agreement between the two figures. 
To conclude, the Adjustment of the diode laser was achieved and some qualitative results were presented. This step is important for further Experiments, which could deliver more quantitative results.
