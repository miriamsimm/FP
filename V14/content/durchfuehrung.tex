\section{Durchführung}
In diesem Versuch sollen die Bestandteile eines Würfels, welcher aus $27$ Elementarwürfeln, mit einer Kantenlänge von 1cm , besteht. Der zu Verfügung gestellte $\gamma$-Strahler ist $\ce{137^Cs}$, mit einem Peak $E_{\gamma} = \SI{662}{\k \eV}$. Die Messdauer für jede Messung wurde auf $\SI{240}{\second}$ gestellt, wobei kleine Variationen durch Totzeiten auftreten.
Zunächst wird eine Leermessung durchgeführt, um eine Zuordnung des $\gamma$-peaks zu ermöglichen. Außerdem wird auch noch ein vollständiges $\gamma$-Spektrum aufgenommen und dokumentiert.
Daraufhin wird ein leeres Aluminiumgehäuse, in welchem sich die Würfel befinden, vermessen, um sich dessen Einfluss auf die Messung klarzumachen.
Anschließend wird der Würfel "3" vermessen, welcher nur aus einem Material besteht. Hierfür werden insgesamt vier Strahlengänge gewählt und die Anzahl der $\gamma$-Teilchen mithilfe eines Programmes ausgelesen.
Zuletzt wird der Würfel "4" vermessen, welcher aus zwei verschiedenen Materialen besteht. Hierfür werden insgesamt zwölf Strahlengänge durchgeführt und erneut die Teilchenanzahl mithilfe des Detektors ausgelesen.