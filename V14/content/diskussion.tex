\section{Diskussion}
Der Verlauf des Spektrums der ${}^{137}\text{Cs}$-Quelle \ref{fig:spektrum} ist wie erwartet, der Photopeak ist eindeutig
erkennbar und auch die Compton-Kante ist gut ersichtlich.
Die Bestimmung des Absorptionskoeffizienten des Materials von Würfel 3 lässt die Schlussfolgerung zu, dass es sich bei diesem
Material um Blei handelt. Die relative Abweichung des experimentell bestimmten Wertes \ref{eq:blei} zum Literaturwert \ref{tab:literatur}
beträgt
\begin{equation*}
    \Delta \mu = \frac{\left|\mu_\text{Exp}-\mu_\text{Lit}\right|}{\mu_\text{Lit}} = 15,7 \% \, .
\end{equation*}
In Tabelle \ref{tab:vergleich} sind die relativen Abweichungen der Absorptionskoeffizienten der Elementarwürfel von Würfel 4 zu den
Literaturwerten aufgeführt, denen sie zugeordnet wurden.
\FloatBarrier
\begin{table}[h]
    \centering
    \caption{Absorptionskoeffizienten der Elementarwürfel von Würfel 4 sowie die relative Abweichung der Werte zu den Literaturwerten 
    der Materialien der Elementarwürfel auf Basis der Zuordnung durch die experimentelle Analyse.}
    \label{tab:vergleich}
    \begin{tabular}{l S[table-format=1.3]@{${}\pm{}$} S[table-format=1.3] l S[table-format=1.3] S[table-format=3.2]}
        \toprule
        {} & \multicolumn{2}{c}{$\mu /\si{\cm}^{-1}$} & {Material} & {$\mu_\text{Lit} /\si{\cm}^{-1}$} & {$\Delta \mu /\%$} \\
        \midrule
        $\mu_1$ & 0.042  & 0.108 & {Delrin} & 0.118 & 64.75 \\ 
        $\mu_2$ & 0.161  & 0.090 & {Delrin} & 0.118 & 36.69 \\
        $\mu_3$ & 0.024  & 0.109 & {Delrin} & 0.118 & 79.81 \\
        $\mu_4$ & 1.030  & 0.095 & {Blei}   & 1.245 & 1.90 \\
        $\mu_5$ & 1.132  & 0.101 & {Blei}   & 1.245 & 7.85 \\
        $\mu_6$ & 1.033  & 0.096 & {Blei}   & 1.245 & 1.61 \\
        $\mu_7$ & 0.139  & 0.108 & {Delrin} & 0.118 & 18.05 \\
        $\mu_8$ & -0.047 & 0.087 & {Delrin} & 0.118 & 139.84 \\
        $\mu_9$ & 0.143  & 0.108 & {Delrin} & 0.118 & 21.00 \\
        \bottomrule
    \end{tabular}
\end{table}
\FloatBarrier
\noindent
Es fällt auf, das besonders die Abweichungen zum Literaturwert von Delrin ($\text{CH}_2\text{O}$) groß sind, 
während die Abweichungen zum experimentell bestimmten Wert von Blei klein sind. Es ist daher interessant zu untersuchen, wie 
groß die Abweichung zu einem experimentell bestimmten Absorptionskoeffizienten von Delrin sind. Die großen Abweichungen sind darüber hinaus
ein deutlicher Indikator dafür, dass es bei der Messung systematische Fehler gab. Die Auffächerung des Strahls ist eine weitere 
mögliche Ursache von Messungenauigkeiten bei der Bestimmung der Absorptionskoeffizienten der Elementarwürfel, da der Strahl auch nebenliegende
Elementarwürfel treffen kann und so die gemessene Intensität beeinflusst. Zudem kann sich auch die Justierung negativ auf die Genauigkeit
der Ergebnisse auswirken, da sie nur per Augenmaß durchgeführt wurde. Für eine präzisere Messung ist es daher besonders 
sinnvoll, die Justierung zu verbessern.