\section{Auswertung}
Abbildung \ref{fig:spektrum} zeigt das Absortpionsspektrumpektrum des ${}^{137}\text{Cs}$-Strahlers.
Zur Erstellung der Abbildungen wird das \texttt{python}-Paket \texttt{matplotlib} \cite{Hunter:2007} verwendet, die Auswertung erfolgt
mit dem Paket \texttt{numpy} \cite{harris2020array}. Die Fehlerrechnung zur Berücksichtigung der Messunsicherheiten wird mit dem Paket \texttt{uncertainties} \cite{uncertainties}
automatisiert.
Über 
\begin{equation*}
    N = \frac{C}{t}
\end{equation*}
wird über die Anzahl der Counts $C$ auf die Zählrate $N$ umgerechnet, die als Maß für die Intensität
verwendet werden kann.
Deutlich erkennbar ist der Photopeak bei Kanalnummer 128, entsprechend der Energie $E= 662 \text{ keV}$ 
und die Compton-Kante im Bereich der Kanalnummern 85 bis 105. 
\begin{figure}
    \centering
    \includegraphics[width = \linewidth]{Leermessung.pdf}
    \caption{Spektrum der ${}^{137}\text{Cs}$-Quelle. Aufgetragen sind die Zählraten der einzelnen
    Kanäle mit den entsprechenden Kanalnummern. Die grüne Linie markiert den Photopeak bei  
    Kanalnummer 128 und der charakteristischen Energie $E_\gamma = 662 \text{ keV}$. Die Compton-
    Kante liegt im Bereich der Kanalnummern 85 bis 105. Erstellt mit dem \texttt{python}-Paket \texttt{matplotlib} \cite{Hunter:2007}.}
    \label{fig:spektrum}
\end{figure}
%TO-DO:Matrix vorher einfügen, Projektionen erklären
Die Absorptionskoeffizienten werden für die Würfel mit den Nummern 1, 3 und 4 bestimmt. Würfel 1 ist leer und besteht nur aus dem Aluminiumgehäuse.
Würfel 3 besteht nur aus einem unbekannten Material, Würfel 4 aus zwei unbekannten Materialien. 
Die gemessenen Counts $C$, Messdauer $t$, Zählraten $N$ sowie die entsprechenden Projektionen der Würfel 3 und 4 
befinden sich in den Tabellen \ref{tab:W3} und \ref{tab:W4}. Die Zählraten $N$ werden mit dem statistischen Fehler $\sqrt{N}$
belegt, die Messdauer wird als fehlerfrei angenommen.
Bei der Untersuchung von Würfel 1 wurden bei einer Messdauer von $241.62 \text{ s}$ 
\begin{equation*}
    C = 37660
\end{equation*}
Counts gemessen.  
Die Berechnung der Absorptionskoeffizienten erfolgt über Gleichung \ref{eq:mu}. Über die Kovarianzmatrix $V$ ergeben sich die entsprechenden 
Unsicherheiten $\sigma_I$ als Wurzeln der Diagonalelemente.
\FloatBarrier
\begin{table}[h]
    \centering
    \caption{Würfel 3, Schweres Material NOCH ÄNDERN}
    %\sisetup{table-format=1.2}
    \label{tab:W3}
    \begin{tabular}{c S[table-format=4.0] S[table-format=3.2] S[table-format=1.2]@{${}\pm{}$} S[table-format=1.2]}
        \toprule
        {Projektion} & {$C$} & {$t/\si{\s}$} & \multicolumn{2}{c}{$N/\si{\s}^{-1}$} \\
        \midrule
        $I_2$ & 1301 & 241.24 & 5.39 & 2.32 \\
        $I_7$ & 1164 & 241.24 & 4.83 & 2.20 \\
        $I_8$ & 594  & 241.24 & 2.46 & 1.57 \\
        \bottomrule
    \end{tabular}
\end{table}
\FloatBarrier
\noindent
\FloatBarrier
\begin{table}[h]
    \centering
    \caption{Würfel 3, Schweres Material NOCH ÄNDERN}
    %\sisetup{table-format=1.2}
    \label{tab:W3}
    \begin{tabular}{c S[table-format=4.0] S[table-format=3.2] S[table-format=1.2]@{${}\pm{}$} S[table-format=1.2]}
        \toprule
        {Projektion} & {$C$} & {$t/\si{\s}$} & \multicolumn{2}{c}{$N/\si{\s}^{-1}$} \\
        \midrule
        $I_2$ & 1301 & 241.24 & 5.39 & 2.32 \\

        \bottomrule
    \end{tabular}
\end{table}
\FloatBarrier
\noindent
