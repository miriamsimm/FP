\section{Auswertung}
Abbildung \ref{fig:Spektrum} zeigt das Absortpionsspektrumpektrum des ${}^{137}\text{Cs}$-Strahlers.
Über 
\begin{equation*}
    N = \frac{C}{t}
\end{equation*}
wird über die Anzahl der Counts $C$ auf die Zählrate $N$ umgerechnet, die als Maß für die Intensität
verwendet werden kann.
Deutlich erkennbar ist der Photopeak bei Kanalnummer 128, entsprechend der Energie $E= 662 \text{ keV}$ 
und die Compton-Kante im Bereich der Kanalnummern 85 bis 105. 
\begin{figure}
    \centering
    \includegraphics[width = \linewidth]{Leermessung.pdf}
    \caption{Spektrum der ${}^{137}\text{Cs}$-Quelle. Aufgetragen sind die Zählraten der einzelnen
    Kanäle mit den entsprechenden Kanalnummern. Die grüne Linie markiert den Photopeak bei  
    Kanalnummer 128 und der charakteristischen Energie $E_\gamma = 662 \text{ keV}$. Die Compton-
    Kante liegt im Bereich der Kanalnummern 85 bis 105.}
    \label{fig:spektrum}
\end{figure}
%TO-DO:Matrix vorher einfügen, Projektionen erklären
Nun werden die Absorptionskoeffizienten der Würfel berechnet. Untersucht
werden die Würfel mit den Nummern 1, 3 und 4. Würfel 1 ist leer und besteht nur aus dem Aluminiumgehäuse.
Würfel 3 besteht nur aus einem unbekannten Material, Würfel 4 aus zwei unbekannten Materialien. 
Die gemessenen Zählraten und Zähldauern sowie die entsprechenden Projektionen der Würfel 3 und 4 
befinden sich in den Tabellen \ref{tab:W3} und \ref{tab:W4}. 
Bei der Untersuchung von Würfel 1 wurden bei einer Messdauer von $241.62 \text{ s}$ $N = 37660$ Counts
gemessen.
\FloatBarrier
\begin{table}[h]
    \centering
    \caption{Würfel 3, Schweres Material NOCH ÄNDERN}
    \sisetup{table-format=4.0}
    \label{tab:W3}
    \begin{tabular}{l c c}
        \toprule
        {Projektion} & {Messdauert $t/\si{\s}$} & {Counts $N$}\\
        \midrule
        $I_2$ & 241.24 & 1301 \\
        $I_5$ & 241.38 & 1300 \\
        $I_7$ & 241.24 & 1164 \\
        $I_8$ & 241.24 & 594 \\
        \bottomrule
    \end{tabular}
\end{table}
\FloatBarrier
\noindent
\FloatBarrier
\begin{table}[h]
    \centering
    \caption{Würfel 4, Materialmischung NOCH ÄNDERN}
    \sisetup{table-format=4.0}
    \label{tab:W4}
    \begin{tabular}{c c c}
        \toprule
        {Projektion} & {Messdauert $t/\si{\s}$} & {Counts $N$}\\
        \midrule
        $I_2$ & 241.24 & 1301 \\
        $I_5$ & 241.38 & 1300 \\
        $I_7$ & 241.24 & 1164 \\
        $I_8$ & 241.24 & 594 \\
        \bottomrule
    \end{tabular}
\end{table}
\FloatBarrier
\noindent