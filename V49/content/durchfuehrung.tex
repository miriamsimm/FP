\section{Versuchsaufbau}
Die Probe steckt in einem bereits präperierten Versuchsaufbau, welcher Spulen zum erzuegen des Magnetfeldes und der Radioimpulse benutzt.
Das Magnetfeld sowie dessen Gradienten kann am Gerät eingestellt werden. 
Des Weiteren können Pulsfolgen bestehend aus einem $A$-puls und bis zu $N$ $B$-Pulsen frei eingestellt werden.
Hier kann die Pulsfrequenz, die Phase, die Pulslänge und der Abstand zwischen zwei Pulsen $\tau$ eingestellt werden.
Das ausgegebene Signal kann auf einem mehr-Kanal-Oszilloskop sichtbar gemacht werden.
\section{Durchführung}
\subsection{Justage}
Zunächst wird der Aufbau mit einer Kalibrationsprobe, welche aus Wasser mit Kupfersulfat besteht, justiert.
Zunächst und zwischen allen Schritten wird jeweils die Temperatur innerhalb der Spule gemessen, da die Temperatur Einfluss auf das Experiment hat.
Es wird eine Periode von $\SI{0.5}{\second} $ eingestellt und es werden $\SI{2}{\micro\second}$ Impulse benutzt, um den freien Induktionsfall auf dem Oszilloskop abzubilden.
Nun wird die Frequenz ausgehend von $\SI{21}{\MHz}$ variiert, bis möglichst wenig Schwingungen zu sehen sind. Anschließend wird die Phase so eingestellt, dass einer der Kanäle kein Signal mehr anzeigt.
Die $90°$ Pulslänge wird so justiert, dass das Signal maximiert wird und die $180°$ Pulslänge ist dann das Doppelte.
\subsection{Messung von $T_1$}
Nun wird die Wasserprobe in den Aufbau eingesetzt, die Pulslänge von A wird auf die $180°$ Pulslänge eingestellt und B auf die $90°$ Pulslänge. Die Periode wird auf $ P = \SI{10}{\second} > 3 T_1$ gestellt und der Pulsabstand $\tau $ variiert. 
Ab Werten von $\tau >\SI{1}{\second}$ wird die Periode auf $tau + \SI{10}{\second}$ gestehlt. In jedem Schritt werden die Messwerte notiert.
\subsection{Messung von $T_2$}
Der A puls wird auf $90°$, und $100$ B-pulse auf die $180°$ Pulslänge gestellt.
Die Periode wird auf $P = \SI{10}{\second}$ gestellt und der "MG" auf "on" gestellt, damit der $90°$-Puls um $90°$ gegenüber dem $180°$ Puls eingestellt ist.
Mithilfe des Oszilloskopes wird ein Bild und eine CSV datei gespeichert und anschließend wird dieser Schritt wiederholt, wobei der "MG" Schalter auf "off" gestellt wird.
\subsection{Difussionsmessung}
Zunächst wird der Gradient entlang der $z$-Achse auf den Maximalwert gestellt. 
Der A puls bleibt auf der $90°$ Länge und der B-Puls bei $180°$, wobei die Anzahl der B-Pulse auf $1$ gesetzt wird.
Die Periode bleibt auch bei $ P = \SI{10}{\second}$ und die Pulslänge $\tau$ variiert, wobei die $\tau$-Werte und die Höhe des echos notiert wird.
Für einen Wert von $\tau$ wird auch ein Bild und eine CSV-Datei mithilfe des Oszilloskopes gespeichert.