\section{Diskussion}
Die Auswertung der Messdaten zur Bestimmung der Spin-Gitter-Relaxationszeit $T_1$ und Spin-Spin-Relaxationszeit 
$T_2$ führt auf
\begin{align*}
    T_1 &= \SI{2.39(8)}{\second} \\
    T_2 &= \SI{1.37(7)}{\second} \, .
\end{align*}
Das Verhältnis dieser Werte ist 
\begin{equation*}
    \frac{T_1}{T_2}_\text{Exp} = \num{1.74(11)} \, .
\end{equation*}
Wie in der Literatur \cite{MGL} ist der gemessene Wert für $T_1$ größer als der Wert für $T_2$.
In Ref. \cite{Chang} findet sich das Verhältnis 
\footnote[1]{Wird das Verhältnis mit den Ergebnissen in Tabelle 1 (Erste Zeile) aus Ref. \cite{Chang}
direkt berechnet, so kann dieses Ergebnis nicht reproduziert werden, stattdessen berechnet sich das Verhältnis
zu $T_1 / T_2 = \num{2.03(16)}$.}
\begin{equation*}
    \left(\frac{T_1}{T_2}\right)_\text{Lit} =1,85 \, .
\end{equation*}
Die Abweichung des experimentell bestimmten Verhältnisses vom Literaturwert beträgt $-5,95 \, \%$.
Für die einzelnen experimentell bestimmten Werte für $T_1$ und $T_2$ ergibt sich bei
Literaturwerten $T_1 = \num{3.09(15)}$ \cite{Chang} und 
$T_2 = \num{1.52(93)}$ \cite{Chang} eine Abweichung von $-22,65 \, \%$ und $-9,87 \, \%$. Das Ergebnis der 
$T_1$-Bestimmung weist also eine deutlich größere Abweichung vom Literaturwert auf als das Ergebnis der 
$T_2$-Bestimmung. Dies ist auch in der Ausgleichsrechnung ersichtlich: Die Ausgleichsfunktion bei der $T_2$-Bestimmung 
in Abb. \ref{fig:T_2} passt deutlich besser an die Messdaten als bei der $T_1$-Bestimmung, vgl. Abbildung \ref{fig:T_1}.
Eine mögliche Ursache dieser Abweichungen ist die Temperaturabhängigkeit der Relaxationszeiten. 
Die Messung wurde wie in der Literatur \cite{Chang}, \cite{MGL} bei Raumtemperatur
durchgeführt, jedoch wurde die Temperatur nicht von außen gesteuert und kann so die Messung beeinflusst haben.
So zeigen insbesondere die Ergebnisse in Ref. \cite{Krynicki} eine deutliche Abhängigkeit von $T_1$ von der
Temperatur. Der hier bestimme Wert für $T_1$ weist die geringste Abweichung zum $T_1$-Wert aus \cite{Krynicki}
bei $10 \, \si{\celsius}$ auf. Die Temperaturmessungen während des Experiments, aufgeführt in Tabelle \ref{tab:temperaturen}, zeigen jedoch deutlich, 
dass Temperaturschwankungen während der Messung keine ausreichende Erklärung für die starke Abweichung von den 
Literaturwerten sind. 
\FloatBarrier
\begin{table}[h]
    \centering
    \caption{Gemessene Temperaturen in der Apparatur zu verschiedenen Zeitpunkten des Experiments.}
    \label{tab:temperaturen}
    \begin{tabular}{S[table-format=2.1] l}
        \toprule
        {$T /\si{\celsius}$} & {Zeitpunkt} \\
        \midrule
        21.7 & {Vor der Justierung} \\
        21.1 & {Nach der Justierung} \\
        21.5 & {Nach der $T_1$-Messung} \\
        22.5 & {Nach der Diffusionsmessung} \\ 
        \bottomrule
    \end{tabular}
\end{table}
\FloatBarrier
\noindent
Ein weiterer Grund für die Abweichungen der Ergebnisse von den Literaturwerten kann außerdem der Versuchsaufbau,
insbesondere die verwendete Probe, sein. Die untersuchte Probe aus bidestilliertem Wasser ist in ein 
Glasrörchen der Dicke $D = 4,2 \, \text{mm}$ eingeschmolzen. Glas ist hydrophil und wechselwirkt über die 
Van der Waals-Wechselwirkung stark mit Wasser, sodass die Wassermoleküle in der Nähe der Probeninnenwand
eine höhere Ordnung aufweisen. Diese Ordnung der Wassermoleküle kann zu verringerten Relaxationszeiten führen \cite{Chang}.
Auf diese Weise können die Abmessungen des Probenröhrchens das Ergebnis beeinflussen. In der Vergleichsliteratur
\cite{Chang} wird zu den Abmessungen der verwendeten Probe keine Angabe gemacht, sodass diese Werte möglicherweise
nur bedingt mit den hier erlangten Resultaten vergleichbar sind. 
Da aber die Vergleichswerte für $T_1$ bei $T = 20 \, \si{\celsius}$ aus den Referenzen \cite{Chang}, \cite{Krynicki}
untereinander nur geringfügig verschieden sind, ist die Auswirkung dieses Effekts als weniger bedeutsam einzuschätzen.
Desweiteren können jedoch auch eine fehlerhafte Justierung des Versuchsaufbaus und ungenaue Messungen zu 
systematischen Messunsicherheiten führen und so die Genauigkeit der Ergebnisse beeinflussen.
Nicht betrachtet wird zudem die Abhängigkeit von $T_2$ vom pH-Wert aufgrund von Wasserstoffbrückenbindungen \cite{MGL}.

Die experimentelle Bestimmung der Diffusionskonstante führt auf 
\begin{equation*}
    D = (1,71 \pm 0,04) \cdot 10^{-9} \, \text{m}^2/\text{s} \, .
\end{equation*}
Der Literaturwert ist $D = (2,78 \pm 0,035) \cdot 10^{-9} \, \text{m}^2/\text{s}$  \cite{Chang}, 
der experimentell bestimmte Wert weicht um $-38,49 \, \%$ vom Literaturwert ab.
Als mögliche Ursache dieser Abweichung muss die Spin-Spin-Relaxationszeit in Betracht gezogen werden, die 
hinreichend groß sein muss, damit die Messung zu sinnvollen Ergebnissen führt. Dies ist gewährleistet wenn 
\begin{equation*}
    T_2^3 \gg \frac{3}{2 D \gamma^2 g^2} \, .
\end{equation*}
In diesem Experiment ist 
\begin{align*}
    T_2^3 &= \num{2.57(39)} \, \text{s}^3 \\
    \frac{3}{2 D \gamma^2 g^2} &= \num{1.642(38)} \cdot 10^{-6} \text{s}^3 \, .
\end{align*}
Das Verhältnis beider Größen ist
\begin{equation*}
    \frac{T_2^3}{\frac{3}{2 D \gamma^2 g^2}} \sim 10^7 \gg 1 \, ,
\end{equation*}
sodass diese Bedingung erfüllt ist.
Auch hier müssen Messungenauigkeiten als Fehlerquelle in Betracht gezogen werden, besonders auch 
in der Bestimmung der Diffusionszeit $T_D$, die in die Berechnung der Diffusionskonstante eingeht. 
Erwartet wird, dass die Messwerte der Echohöhen auf einer Gerade liegen, vgl. unterer Plot in Abbildung \ref{fig:echo}.
Diese Erwartung wird in guter Näherung erfüllt.
Jedoch ist zu beachten, dass in der Ausgleichsrechnung zur Bestimmung von $T_D$ der zuvor ermittelte Wert $T_2$
verwendet wird. Wie oben beschrieben, weicht dieser Wert jedoch deutlich von Vergleichwerten 
in der Literatur ab und kann so das Ergebnis der Diffusionsmessung beeinflussen. Zudem ist auch die Bestimmung des Durchmessers 
$d_f$, der über die Gradientenstärke $g$ quadratisch in die Formel zur Bestimmung des 
Diffusionskonstanten eingeht, durch reines Ablesen aus der Fouriertransformation höchst ungenau und sollte 
in Folgemessungen verbessert werden. 
Es wäre außerdem interessant, bei der Bestimmung der Diffusionszeit
in der Ausgleichsrechnung einen Literaturwert für $T_2$ zu verwenden und so analysieren zu können, wie 
die Unsicherheiten bei der $T_2$-Messung das Ergebnis für die Diffusionskonstante beeinflussen.