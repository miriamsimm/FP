\section{Zielsetzung}
In diesem Versuch wird das Relaxionsverhalten der Spin-Gitter- und Spin-Spin-Wechselwirkungen einer Wasserprobe untersucht.
Anschließend wird das Diffusionsverhalten von Wasser untersucht.
\section{Theorie}
Atomkerne haben einen Gesamtdrehimpuls $\vec{J}$, welcher sich aus dem Spin $\vec{S}$ und dem Bahndrehimpuls $\vec{L}$ zusammensetzt.
Ein Teilchen mit Drehimpuls induziert auch ein magnetisches Moment $\vec{\mu}$. Der Proportionalitätsfaktor ist durch das gyromagentische Verhältnis $\gamma$ gegeben.
Im folgenden Versuch wird Wasser untersucht, was hauptsächlich aus Wasserstoffkernen mit $ j = \frac{1}{2}$ besteht.
Hierfür ist das gyromagentische Verhältnis gegeben durch \cite{SciPy}
\begin{equation*}
    \gamma = 267,52 \, \frac{1}{\text{sT}} \, .
\end{equation*} 
Bei Anlegen eines magnetischen Feldes $\vec{B}$ richten sich die einzelnen magnetischen Momente $\vec{\mu_i}$ an dem äußeren Magnetfeld aus, da dies energetisch günstiger ist.
Dies führt zu einer makroskopischen Magnetisierung 
\begin{equation*}
    \vec{M} = \sum_i \vec{\mu}_i.
\end{equation*}
Das wirkende Drehmoment führt zu einer Rotation des Spins um die Achse des Magnetfeldes. Dies geschieht mit einer Winkelgeschwindigkeit $\omega_L = \gamma |\vec{B_0}|$, welche auch als Lamorfrequenz bezeichnet wird.
Wird nun ein weiteres Magnetfeld $\vec{B}_1 $ orthogonal zu $\vec{B}_0$ für eine Zeit $t_p$ hinzugeschaltet, beginnt der Spin resonant zu präzedieren. 
Der Spin wird dabei um einen Winkel $\theta = \gamma |\vec{B_1}| t_p$ abgelenkt. Anschließend relaxiert der Spin durch zwei Effekte, was durch die Bloch'schen Gleichungen 
\begin{align}
    \frac{\mathrm{d}Mx}{\mathrm{d}t} &= \gamma B_0 M_y - \frac{M_x}{T_2} \\  
    \frac{\mathrm{d}My}{\mathrm{d}t} &= -\gamma B_0 M_x - \frac{M_y}{T_2} \\   
    \frac{\mathrm{d}Mz}{\mathrm{d}t} &=  \frac{M_0 - M_z}{T_2} 
\end{align}  
beschrieben wird.
Hierbei ist $M_i $ die Magnetisierung in die jeweilige Richtung und $T_i$ sind Relaxationszeiten, auf die im folgenden genauer eingangen werden soll.
\subsection{Spin-Gitter-Relaxation}
Die Energie eines Dipols wird minimal, wenn diese entlang des $\vec{B}$-Feldes ausgerichtet sind. Thermische Schwankungen führen dazu, dass die durch Radiowellenpulse angeregten Spins wieder entlang der $z$-Achse relaxieren. Hierbei wird Energie an das Gitter abgegeben, weshalb dies als Spin-Gitter-Relaxation bezeichnet wird. Dies wird durch die Zeitkonstante $T_1$ charakterisiert.
Die longitudinale Magnetisierung ist gegeben durch
\begin{equation}
    M_{z} (\tau) = M_0 \left( 1- 2 \exp(-\frac{\tau}{T_1} ) \right) ,
\end{equation}
wobei $M_0$ die Startmagnetisierung ist.
\subsection{Spin-Spin-Relaxation}
Anregung durch einen $90°$ Impuls in der $xy$-Ebene führt zu einem weiteren Relaxationsprozess, welcher durch $T_2$ charakterisiert ist.
Dies geschieht, weil das äußere Magnetfeld immer kleine Inhomogenitäten aufweist, welche zu einer Dephasierung der präzedierenden Kernspins führt.
Das sich die Kernspins hier gegenseitig beeinflussen, wird dies als Spin-Spin-Relaxation bezeichnet.
Es ergibt sich ein Exponentialgesetz 
\begin{equation}
    \label{eqn:T2}
    M_{x,y}(\tau) = M_0 \exp\left(\frac{-2\tau}{T_2} \right) 
\end{equation}
für die transversale Magnetisierung.
\subsection{Diffusionsverhalten}
Brownsche Molekularbewegung führt dazu, dass die Probe diffundiert und ermöglicht es, diesen ortsabhängigen Prozess zu untersuchen.
Hierfür muss ein Feldgradient eingeführt werden, welcher es ermöglicht, die nun ortsabhängige Lamorfrequenz zu messen. 
Dies führt zu einer Modifikation von Formel \ref{eqn:T2} zu
\begin{equation}
    M_{x,y}(2\tau) = M_0 \exp\left(-\frac{2\tau}{T_2}\right) \exp \left(-\frac{\tau^3}{T_D}\right).
\end{equation}
mit der Diffusionszeit 
\begin{equation}
    T_D = \frac{3}{D \gamma^2 G^2 \tau^2} \, .
    \label{eq:diffusionskonstante}
\end{equation}