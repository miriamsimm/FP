\section{Durchführung}
Bevor die eigentliche Messung durchgeführt werden kann, muss der Versuchsaufbau passend eingestellt werden.
Dazu wird zuerst sichergestellt, dass Hochspannung an den PMTs anliegt, indem ein Oszilloskop angeschlossen
wird. Anschließend wird die Schwellspannung beider Diskriminatoren variiert, sodass an beiden Ausgängen
etwa 30 Impulse pro Sekunde gemessen werden. Als Pulsdauer wird $\Delta t = \SI{10}{\nano\second}$ eingestellt.

Dann wird die Koinzidenzschaltung justiert. Dazu wird jeweils eine der beiden Verzögerungsleitungen 
variiert und die Impulsrate am Ausgang der Koinzidenz gemessen. Um die passende Verzögerung zu bestimmen, werden
für beide Leitungen jeweils zehn Messwerte genommen, wobei der Messbereich so gewählt wird, dass die Halbwertsbreite 
der Verteilung der Impulsrate bestimmt werden kann. Aus dieser Messreihe wird als Verzögerung $\Delta t = \SI{1}{\nano\second}$
gewählt. 

Danach wird als Suchzeit am Monoflop $T_\text{s} = \SI{10}{\micro\second}$ eingestellt. Zur Kalibrierung des 
MCA wird ein Doppelimpulsgenerator an die Koinzidenz angeschlossen, sodass der zeitliche Abstand zweier 
Impulse dem entsprechende Kanal des MCA zugeordnet werden kann. Dazu werden elf Messwerte mit Zeitdifferenzen
von $\SI{0.7}{\micro\second}$ bis $\SI{9.9}{\micro\second}$ bei einer Messzeit von $\SI{10}{\second}$ genommen.
Anschließend erfolgt die Lebensdauermessung.