\section{Diskussion}
%Die Kalibration und Justage des Gerätes verlief im Allgemeinen sehr gut. 
%Der gewählte Wert für die Verzögerungszeit $T_{V} = \SI{-1}{\nano\second}$ weich vom
%Erwartungswert $\mu = \SI{0.6(4)}{\nano\second}$ etwas ab. Der eingestellte Wert ist jedoch aussreichend gut 
%gewählt, da dieser trotzdem auf dem Plateau\ref{fig:Verzögerung} liegt und diese Genauigkeit hinreichend ist.
Ein breites Plateau, welches bei der Justage bestimmt worden ist, ist in Abbildung \ref{fig:Verzögerung} gut erkennbar. Die eingestellte Verzörgerungszeit von $T_V = \SI{-1}{\second}$ liegt etwas Links vom Zentrum des Plateaus.
Für die optimale Verzögerung wäre ein Wert in Zentrum besser, damit die Eingänge der Koinzidenzschaltung das selbe Ergebnis messen. Die Bestimmung der Halbwertsbreite war auch möglich, wobei kein Vergleichswert vorliegt.
Der lineare Fit, welcher die Zeiten der Kanalnummer zuordnet, verlief ebenfalls wie erwartet. Es gibt einen 
abweichenden Wert in Abbildung \ref{fig:kali}, sonst ist der lineare Zusammenhang jedoch nahezu ideal erfüllt. 
Die Abweichung kann dadurch erklärt werden, dass die Kanalnummer aus einem Programm abgelesen 
wird, wobei ein Peak mit endlicher Breite abgelesen werden musste. 
Die Bestimmung der Lebensdauer von $\tau = \SI{2.42(5)}{\micro\second}$ hat eine Abweichung von $\SI{10}{\percent}$ zum 
Literaturwert $\tau_{\text{Lit}} = \SI{2.20}{\micro\second}$ \cite{PDG}. Die Abweichung kann 
dadurch erklärt werden, dass die Messung der Lebensdauer in Materie stattgefunden hat. In Materie können 
sich unter anderem Myonische Atome bilden, welche zusätzliche Lichtblitze erzeugen.
Eine Bestimmung im Vakuum, bzw. in weniger dichten Materialen, könnte also zu besseren Ergebnisse 
führen. Des weiteren hat der Vielkanalanlysator auch eine endliche Auflösung, welche zu Falschzuordnungen von einzelnen Events führen kann.

% Desweiteren kann auch die eingestellte Verzögerungszeit das Ergebnis beeinflussen. 
%Hierbei ist auch anzumerken, dass die Schalter zum Einstellen der Verzögerungszeit nicht immer reagiert 
%haben. Teilweise mussten die Schalter öfters betätigt werden um sicherzustellen, dass die 
%Verzörgerungszeit richtig eingestellt war. Vollständig auszuschließen, dass ein anderer Wert 
%eingestellt war, ist daher ebenfalls nicht möglich.