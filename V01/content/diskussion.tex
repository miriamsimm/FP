\section{Diskussion}
%Die Kalibration und Justage des Gerätes verlief im Allgemeinen sehr gut. 
%Der gewählte Wert für die Verzögerungszeit $T_{V} = \SI{-1}{\nano\second}$ weich vom
%Erwartungswert $\mu = \SI{0.6(4)}{\nano\second}$ etwas ab. Der eingestellte Wert ist jedoch aussreichend gut 
%gewählt, da dieser trotzdem auf dem Plateau\ref{fig:Verzögerung} liegt und diese Genauigkeit hinreichend ist.
%Ein breites Plateau, welches bei der Justage bestimmt worden ist, ist in Abbildung \ref{fig:Verzögerung} gut erkennbar. 
Für die eingestellte Verzörgerungszeit von $T_V = \SI{-1}{\nano\second}$, um deren Wert das Plateau 
in Abbildung \ref{fig:Verzögerung} bei optimaler Einstellung zentriert sein sollte, ist das 
in der Justage bestimmte Plateau etwas nach rechts verschoben. Bei einer erneuten Messung sollte daher 
eine Verzögerungszeit von etwa $\SI{1}{\nano\second}$ eingestellt werden, um zu gewährleisten,
dass die Eingänge der Koinzidenzschaltung das gleiche Ergebnis messen. 

%Die eingestellte Verzörgerungszeit von $T_V = \SI{-1}{\nano\second}$ liegt etwas links vom Zentrum des Plateaus, siehe Abbildung \ref{fig:Verzögerung}, welches in der Justage bestimmt worden ist.
%Für die optimale Verzögerung wäre ein Wert in Zentrum besser, damit die Eingänge der Koinzidenzschaltung das selbe Ergebnis messen. 
Die Halbwertsbreite $\Delta T_{1/2} = \SI{16.5(14)}{\nano \second}$ hat eine Abweichung von $\SI{65}{\percent} $ zur voreingestellten Impulsbreite $\Delta t = \SI{10}{\nano\second}$. 
Ein möglicher Faktor, der zu dieser Abweichung geführt haben könnte, ist, dass die Schalter zum Einstellen der Verzörgerungszeit nicht immer fehlerfrei reagiert haben. 
%Dies lässt sich dadurch erklären, dass die Schalter zum Einstellen der Verzörgerungszeit nicht immer reagiert haben. 
Es kann daher nicht ausgeschlossen werden, dass die Werte für $t$ in Tabelle \ref{tab:Verzögerung} nicht mit 
den tatsächlich am Gerät eingestellten Werten übereinstimmen.
%Es kann daher nicht vollständig ausgeschlossen werden, dass die Werte für $\Delta t $ in \ref{tab:Verzögerung} tatsächlich am Gerät eingestellt waren.
Der lineare Fit, welcher die Zeiten der Kanalnummer zuordnet, zeigt ebenfalls den erwarteten Verlauf. Es gibt einen 
abweichenden Wert in Abbildung \ref{fig:kali}, sonst ist der lineare Zusammenhang jedoch nahezu perfekt gegeben. 
Als Ursache für die beschriebene Abweichung müssen Fehler in der Versuchsdurchführung in Betracht gezogen werden.
Die jeweiligen Kanalnummern wurden aus den Peaks des vom Vielkanalanalysator
erstellten Histogramms abgelesen. Da diese Peaks eine endliche Breite aufweisen, ist es daher möglich, 
dass die Kanalnummer des beschriebenen Werts nicht richtig zugeordnet wurde.

%Der abweichende Wert kann dadurch erklärt werden, dass die Kanalnummer aus einem Programm abgelesen wird. 
Die Bestimmung der Lebensdauer von $\tau = \SI{2.42(5)}{\micro\second}$ hat eine Abweichung von $\SI{10}{\percent}$ zum 
Literaturwert $\tau_{\text{Lit}} = \SI{2.20}{\micro\second}$ \cite{PDG}. Die Abweichung kann 
dadurch erklärt werden, dass die Messung der Lebensdauer in Materie stattgefunden hat. In Materie können 
sich unter anderem Myonische Atome bilden, welche zusätzliche Lichtblitze erzeugen, die dann 
fehlerhaft detektiert werden.
Eine Bestimmung im Vakuum oder in weniger dichten Materialen könnte also zu besseren Ergebnissen 
führen. Desweiteren hat der Vielkanalanlysator auch eine endliche Auflösung, welche zu Falschzuordnungen von einzelnen Events führen kann.

% Desweiteren kann auch die eingestellte Verzögerungszeit das Ergebnis beeinflussen. 
%Hierbei ist auch anzumerken, dass die Schalter zum Einstellen der Verzögerungszeit nicht immer reagiert 
%haben. Teilweise mussten die Schalter öfters betätigt werden um sicherzustellen, dass die 
%Verzörgerungszeit richtig eingestellt war. Vollständig auszuschließen, dass ein anderer Wert 
%eingestellt war, ist daher ebenfalls nicht möglich.