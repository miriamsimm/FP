\section{Diskussion}
Die Kalibration und Justage des Gerätes verlief im Allgemeinen sehr gut. Der gewählte Wert für die Verzögerungszeit $T_{V} = \SI{-1}{\nano\sec}$ etwas entfernt, von dem Erwartungswert $\mu = \SI{0.6(4)}{\nano\sec}$. Der eingestellte Wert war jedoch aussreichend gut gewählt,da dieser immernoch auf dem Plateau\ref{fig:Verzögerung} liegt und die Genauigkeit nicht so wichtig ist.
Der lineare Fit, welcher die Zeiten der Kanalnummer zuordnet, verlief auch gut. Es gibt einen Wert in Abbildung \ref{fig:kali} welcher abweicht, aber sonst ist der lineare Zusammenhang fast perfekt. Der Abweichende Wert kann dadurch erklärt werden, dass die Kanalnummer aus einem Programm abgelesen wurde musste. Hier musste ein Peak, mit einer endlichen Breite, abgelesen werden. 
Die Bestimmung der Lebensdauer von $\tau = \SI{2.42(5)}{\micro\second}$ hat eine Abweichung, zum Literaturwert \cite{PDG}, $\tau_{Lit} = \SI{2.20}{\micro\second}$, von $10 \%$. Die Abweichung kann dadurch erklärt werden, dass die Messung der Lebensdauer in Materie stattgefunden hat. hier können sich unter anderem Myonische Atome bilden, welche zusätzliche Lichtblitze erzeugen.
Eine Bestimmung im Vakuum, bzw. in weniger dichten Materialen, könnte also bessere Ergebnisse liefern. Des Weiteren kann auch die eingestellte Verzögerungszeit Einfluss auf das Ergebnis haben. Hier ist auch anzumerken, dass die Schalter, zum einstellen der Verzögerungszeit nicht immer reagiert haben. Teilweise mussten die Schalter öfters betätigt werden um sicherzustellen, dass die Verzörgerungszeit richtig eingestellt war. Vollständig auszuschließen, dass ein anderer Wert eingestellt war ist jedoch nicht möglich.