\section{Auswertung}
\subsection{Justage und Kalibration}
Zunächst muss die optimale Verzögerungszeit $T_v$ bestimmt werden. Hierfür werden in einem $10 \text{ s}$-Intervall die Myonen-Counts aufgenommen, wobei die Verzögerung an den Diskriminatoren jeweils verstellt wird. Die aufgenommenen Messwerte sind in Tabelle \ref{tab:Verzögerung} eingetragen.
\begin{table}
    \centering
    \caption{Messwerte zur bestimung der Verzögerungszeit. Für die Durchführung des Versuches wurde der Wert $T_V = \SI{-1}{\micro\second}$ gewählt. Die negative Zeitwerte stehen für eine Verzögerung am linken Diskriminator}
    \begin{tabular}{S[table-format=3.1] S[table-format=3.0]}
    \toprule
    {$t / \si{\nano\second}$} & Counts  \\ 
    \midrule
-14  & 12 \\
-12  & 20 \\
-8   & 89 \\
-4   & 199\\
-3   & 218\\
-2.5 & 210\\
-2   & 233\\
-1.5 & 237\\
-1   & 215\\
-0.5 & 243\\
0    & 227\\ 
0.5  &164\\
1    &258\\
1.5  &197\\
2    &240\\
2.5  &272\\
3    &204\\
4    &235\\
6    &170\\
8    &119\\
12   &40\\
\bottomrule
    \end{tabular}
    \label{tab:Verzögerung}
\end{table} 
Im Folgenden wird die Halbwertsbreite bestimmt, wofür eine Gaussfunktion
\begin{equation*}
    f(t) = A\exp(- \frac{(t-\mu)^2}{2\sigma^2} )
\end{equation*}
verwendet wird. Die Fitwerte und die berechnete Halbwertsbreite $T_{1/2} =2\sqrt{2\ln2}\sigma$ sind gegeben durch 
\begin{align*}
    A &= \num{237(7)} \\
    \mu &= \SI{0.6(4)}{\nano\second} \\
    \sigma &= \SI{6.4(5)}{\nano\second} \\
    T_{1/2} &= \SI{15.1(12)}{\nano\second}.
\end{align*}
Die Messwerte und Fitkurve sind in Abbildung \ref{fig:Verzögerung} dargestellt. Die negative Zeitwerte stehen für eine Verzögerung am linken Diskriminator. 
Für die nachfolgenden Auswertungsschritte wird jeweils eine Verzögerungszeit von $T_v = \SI{-1}{\second}$ gewählt.
\begin{figure}
    \centering
    \includegraphics[width = \linewidth]{plot.pdf}
    \caption{Messdaten zur Justage der Verzögerungszeit an den Diskriminatoren. Aufgetragen sind die Counts in Abhängigkeit von der eingestellten Verzögerungszeit. Negative Zeiten stehen für eine Verzögerung am linken Diskriminator. Zusätzlich ist auch der bestimmte Gauss-Fit abgebildet.}
    \label{fig:Verzögerung}
\end{figure} 
Als nächstes wird die Kalibration durchgeführt. Die Messzeiten betragen erneut $\SI{10}{\second}$ und die Fitgerade für die Umrechnung der Kanalnummern in Messzeiten kann durch 
\begin{equation}
    \label{eqn:kali}
    t(C) = AC + B
\end{equation}
parametrisiert werden. Hierbei ist $t$ die Messzeit, $C$ die Kanalnummer und $A,B$ sind die freien Fitparameter welche durch
\begin{align*}
    A &= \SI{0.0437(13)}{\micro\second} \\
    B &= \SI{0.43(17)}{\micro\second}
\end{align*}
gegeben sind. Die Messwerte für die Kalibration sind in \ref{tab:kali} angegeben und in Abbildung \ref{fig:kali} mit der Fitgerade dargestellt.
Im Folgenden wird allen Kanalnummern mithilfe von Gleichung \ref{eqn:kali} die jeweilige Zeit zugeordnet.
\begin{figure}
    \centering
    \includegraphics[width = \linewidth]{kalibration.pdf}
    \caption{Messwerte zur Kalibration des MCA und die berechnete Fitgerade. Aufgetragen ist die Zeit $t$ in Abhängigkeit von der Kanalnummer.}
    \label{fig:kali}
\end{figure}
\begin{table}
    \centering
    \caption{ Messwerte zur Kalibration des MCA. Die Werte erlauben durch einen linearen Fit eine Zuordnung der Channels an die jeweilige Zeit $t$.}
    \begin{tabular}{S[table-format=3.0] S[table-format=1.1]}
    \toprule
     {Channel} & $t / \si{\micro\second}$  \\ 
     \midrule
     9  & 0.7 \\
     29 & 1.6 \\
     49 & 2.5 \\
     70 & 3.4 \\
     96 & 4.5 \\
     95 & 5.4 \\
     136 & 6.3 \\
     157 & 7.2 \\
     178 & 8.1 \\
     197 & 9.0 \\
     217 & 9.9 \\ 
     \bottomrule
    \end{tabular}
    \label{tab:kali}
\end{table} 
\subsection{Messung der Lebensdauer}
Die Messung der Lebensdauer der Myonen erfolgte über einen Zeitraum von $\SI{181285}{\second}$, wobei $3604339$ Startsignale gemessen wurden. 
Der Untergrund lässt sich mithilfe der Poissonverteilung 
\begin{equation*}
    P(n) = \frac{\lambda^k e^{-\lambda}}{k!}
\end{equation*}
bestimmen. Hier ist $\lambda$ der Erwartungswert der sich aus der Anzahl der Startsignale $N_\text{start}$, der Messzeit $t_\text{Mess}$ und der Suchzeit $T_S = \SI{10}{\micro\second}$ über 
\begin{equation*}
    \lambda = \frac{N_\text{start}}{t_\text{mess}} T_s  
\end{equation*}
bestimmen lässt. Der gesamte Untergrund für genau ein eintreffendes Myon ist dann gegeben durch 
\begin{equation*}
    U = P_{\lambda}(n=1) N_{start} = \num{716}  .
\end{equation*}
Der Untergrund verteilt sich gleichmäßig auf $511$ Channel, weshalb der Untergrund in einem Kanal gegeben ist durch 
\begin{equation*}
    U_c = \frac{U}{511} =\num{1}  .
\end{equation*}
Im Folgenden wird dieser Untergrund von den Messwerten abgezogen.
Die gemessenen Counts, die zugehörige Kanalnummer und die zugeordnete Lebensdauer sind in Tabelle \ref{tab:Lebensdauer} aufgeführt.
Nach dem Zerfallsgesetz \ref{eqn:Zerfall} gilt für den Logarithmus der Counts 
\begin{equation*}
    \ln N = -\frac{t}{\tau} + \ln(N_0).
\end{equation*}
Hierbei ist $\tau $ die gesuchte Lebensdauer und $N_0$ die Anzahl der Myonen für $t=0$. Mit einem linearen Fit an den Logarithmus der Events lassen sich die Parameter bestimmen zu 
\begin{align*}
    \tau &= \SI{2.42(5)}{\micro\second} \\
    N_0 &= \num{207(10)}
\end{align*}
wobei alle Channel mit $0$ Events vernachlässigt werden.
Die Messwerte und die Fitgerade sind in Abbildung \ref{fig:Lebensdauer} logarithmisch aufgetragen.
\begin{figure}
    \centering 
    \includegraphics[width = \linewidth]{lebensdauer.pdf}
    \caption{Messwerte für die Bestimmung der Lebensdauer von Myonen. Aufgetragen sind die Anzahl der Counts in logartihmischer Darstellung gegen die 
    Zeitdifferenz zwischen dem Eintreten in das Medium und dem Zerfall des Myons. 
    Zusätlich abgebildet ist der berechnete lineare Fit.}
    \label{fig:Lebensdauer}
\end{figure}

\begin{table}
    \centering
    \caption{ Aufgenommene Messwerte für die Anzahl der Events in den Channels des MCA. Außerdem ist die zugeordnete Zeit$t$, aus \ref{eqn:kali}, eingetragen. Die Channel $228-511$ hat keine Events und sind deshalb nicht eingetragen }

    \resizebox*{!}{\textheight}{ \begin{tabular}{@{}lll|lll|lll|lll@{}}
    \toprule
     Channel  & Events&$t /  \si{\micro\second}$ &Channel  & Events&$t /  \si{\micro\second}$ &Channel  & Events&$t /  \si{\micro\second}$ &Channel  & Events&$t /  \si{\micro\second}$  \\ \midrule
     0 & 0.4 & 0.0 & 57 & 2.9 & 57.0 & 114 & 5.4 & 20.0 & 171 & 7.9 & 8.0  \\ 
     1 & 0.5 & 0.0 & 58 & 3.0 & 58.0 & 115 & 5.5 & 19.0 & 172 & 7.9 & 10.0  \\ 
     2 & 0.5 & 0.0 & 59 & 3.0 & 56.0 & 116 & 5.5 & 19.0 & 173 & 8.0 & 11.0  \\ 
     3 & 0.6 & 0.0 & 60 & 3.1 & 56.0 & 117 & 5.5 & 22.0 & 174 & 8.0 & 6.0  \\ 
     4 & 0.6 & 134.0 & 61 & 3.1 & 69.0 & 118 & 5.6 & 16.0 & 175 & 8.1 & 4.0  \\ 
     5 & 0.6 & 165.0 & 62 & 3.1 & 55.0 & 119 & 5.6 & 13.0 & 176 & 8.1 & 6.0  \\ 
     6 & 0.7 & 192.0 & 63 & 3.2 & 62.0 & 120 & 5.7 & 11.0 & 177 & 8.2 & 6.0  \\ 
     7 & 0.7 & 185.0 & 64 & 3.2 & 54.0 & 121 & 5.7 & 21.0 & 178 & 8.2 & 12.0  \\ 
     8 & 0.8 & 189.0 & 65 & 3.3 & 63.0 & 122 & 5.8 & 19.0 & 179 & 8.3 & 6.0  \\ 
     9 & 0.8 & 168.0 & 66 & 3.3 & 60.0 & 123 & 5.8 & 20.0 & 180 & 8.3 & 5.0  \\ 
     10 & 0.9 & 154.0 & 67 & 3.4 & 41.0 & 124 & 5.8 & 23.0 & 181 & 8.3 & 9.0  \\ 
     11 & 0.9 & 144.0 & 68 & 3.4 & 53.0 & 125 & 5.9 & 19.0 & 182 & 8.4 & 8.0  \\ 
     12 & 1.0 & 153.0 & 69 & 3.4 & 44.0 & 126 & 5.9 & 13.0 & 183 & 8.4 & 11.0  \\ 
     13 & 1.0 & 158.0 & 70 & 3.5 & 49.0 & 127 & 6.0 & 21.0 & 184 & 8.5 & 10.0  \\ 
     14 & 1.0 & 144.0 & 71 & 3.5 & 53.0 & 128 & 6.0 & 15.0 & 185 & 8.5 & 6.0  \\ 
     15 & 1.1 & 149.0 & 72 & 3.6 & 40.0 & 129 & 6.1 & 15.0 & 186 & 8.6 & 10.0  \\ 
     16 & 1.1 & 161.0 & 73 & 3.6 & 45.0 & 130 & 6.1 & 16.0 & 187 & 8.6 & 7.0  \\ 
     17 & 1.2 & 129.0 & 74 & 3.7 & 45.0 & 131 & 6.2 & 14.0 & 188 & 8.6 & 7.0  \\ 
     18 & 1.2 & 132.0 & 75 & 3.7 & 54.0 & 132 & 6.2 & 13.0 & 189 & 8.7 & 10.0  \\ 
     19 & 1.3 & 139.0 & 76 & 3.8 & 60.0 & 133 & 6.2 & 15.0 & 190 & 8.7 & 6.0  \\ 
     20 & 1.3 & 109.0 & 77 & 3.8 & 50.0 & 134 & 6.3 & 12.0 & 191 & 8.8 & 6.0  \\ 
     21 & 1.3 & 146.0 & 78 & 3.8 & 39.0 & 135 & 6.3 & 13.0 & 192 & 8.8 & 8.0  \\ 
     22 & 1.4 & 115.0 & 79 & 3.9 & 47.0 & 136 & 6.4 & 18.0 & 193 & 8.9 & 5.0  \\ 
     23 & 1.4 & 115.0 & 80 & 3.9 & 47.0 & 137 & 6.4 & 14.0 & 194 & 8.9 & 7.0  \\ 
     24 & 1.5 & 116.0 & 81 & 4.0 & 42.0 & 138 & 6.5 & 11.0 & 195 & 8.9 & 4.0  \\ 
     25 & 1.5 & 107.0 & 82 & 4.0 & 45.0 & 139 & 6.5 & 10.0 & 196 & 9.0 & 8.0  \\ 
     26 & 1.6 & 100.0 & 83 & 4.1 & 40.0 & 140 & 6.5 & 15.0 & 197 & 9.0 & 1.0  \\ 
     27 & 1.6 & 114.0 & 84 & 4.1 & 29.0 & 141 & 6.6 & 11.0 & 198 & 9.1 & 10.0  \\ 
     28 & 1.7 & 119.0 & 85 & 4.1 & 34.0 & 142 & 6.6 & 12.0 & 199 & 9.1 & 6.0  \\ 
     29 & 1.7 & 110.0 & 86 & 4.2 & 42.0 & 143 & 6.7 & 18.0 & 200 & 9.2 & 5.0  \\ 
     30 & 1.7 & 98.0 & 87 & 4.2 & 30.0 & 144 & 6.7 & 14.0 & 201 & 9.2 & 7.0  \\ 
     31 & 1.8 & 119.0 & 88 & 4.3 & 40.0 & 145 & 6.8 & 15.0 & 202 & 9.3 & 8.0  \\ 
     32 & 1.8 & 106.0 & 89 & 4.3 & 27.0 & 146 & 6.8 & 8.0 & 203 & 9.3 & 8.0  \\ 
     33 & 1.9 & 105.0 & 90 & 4.4 & 37.0 & 147 & 6.9 & 19.0 & 204 & 9.3 & 4.0  \\ 
     34 & 1.9 & 88.0 & 91 & 4.4 & 40.0 & 148 & 6.9 & 10.0 & 205 & 9.4 & 8.0  \\ 
     35 & 2.0 & 90.0 & 92 & 4.4 & 37.0 & 149 & 6.9 & 15.0 & 206 & 9.4 & 5.0  \\ 
     36 & 2.0 & 106.0 & 93 & 4.5 & 23.0 & 150 & 7.0 & 8.0 & 207 & 9.5 & 5.0  \\ 
     37 & 2.0 & 85.0 & 94 & 4.5 & 30.0 & 151 & 7.0 & 14.0 & 208 & 9.5 & 11.0  \\ 
     38 & 2.1 & 102.0 & 95 & 4.6 & 31.0 & 152 & 7.1 & 10.0 & 209 & 9.6 & 9.0  \\ 
     39 & 2.1 & 88.0 & 96 & 4.6 & 27.0 & 153 & 7.1 & 7.0 & 210 & 9.6 & 7.0  \\ 
     40 & 2.2 & 97.0 & 97 & 4.7 & 28.0 & 154 & 7.2 & 6.0 & 211 & 9.6 & 7.0  \\ 
     41 & 2.2 & 99.0 & 98 & 4.7 & 29.0 & 155 & 7.2 & 9.0 & 212 & 9.7 & 2.0  \\ 
     42 & 2.3 & 89.0 & 99 & 4.8 & 26.0 & 156 & 7.2 & 13.0 & 213 & 9.7 & 6.0  \\ 
     43 & 2.3 & 85.0 & 100 & 4.8 & 27.0 & 157 & 7.3 & 8.0 & 214 & 9.8 & 7.0  \\ 
     44 & 2.4 & 83.0 & 101 & 4.8 & 19.0 & 158 & 7.3 & 9.0 & 215 & 9.8 & 6.0  \\ 
     45 & 2.4 & 76.0 & 102 & 4.9 & 32.0 & 159 & 7.4 & 13.0 & 216 & 9.9 & 3.0  \\ 
     46 & 2.4 & 89.0 & 103 & 4.9 & 31.0 & 160 & 7.4 & 13.0 & 217 & 9.9 & 3.0  \\ 
     47 & 2.5 & 82.0 & 104 & 5.0 & 23.0 & 161 & 7.5 & 8.0 & 218 & 10.0 & 9.0  \\ 
     48 & 2.5 & 78.0 & 105 & 5.0 & 25.0 & 162 & 7.5 & 10.0 & 219 & 10.0 & 5.0  \\ 
     49 & 2.6 & 66.0 & 106 & 5.1 & 15.0 & 163 & 7.6 & 19.0 & 220 & 10.0 & 6.0  \\ 
     50 & 2.6 & 96.0 & 107 & 5.1 & 24.0 & 164 & 7.6 & 13.0 & 221 & 10.1 & 2.0  \\ 
     51 & 2.7 & 69.0 & 108 & 5.1 & 26.0 & 165 & 7.6 & 10.0 & 222 & 10.1 & 5.0  \\ 
     52 & 2.7 & 68.0 & 109 & 5.2 & 24.0 & 166 & 7.7 & 17.0 & 223 & 10.2 & 4.0  \\ 
     53 & 2.7 & 83.0 & 110 & 5.2 & 22.0 & 167 & 7.7 & 13.0 & 224 & 10.2 & 7.0  \\ 
     54 & 2.8 & 69.0 & 111 & 5.3 & 10.0 & 168 & 7.8 & 7.0 & 225 & 10.3 & 6.0  \\ 
     55 & 2.8 & 67.0 & 112 & 5.3 & 24.0 & 169 & 7.8 & 14.0 & 226 & 10.3 & 6.0  \\ 
     56 & 2.9 & 66.0 & 113 & 5.4 & 18.0 & 170 & 7.9 & 8.0 & 227 & 10.3 & 4.0  \\ 
     \bottomrule
    \end{tabular}}
    \label{tab:Lebensdauer}
\end{table}
