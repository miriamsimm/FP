\section{Myon}
\subsection{Das Myon}
Innerhalb des Standardmodelles der Teilchenphysik ist das Myon ein Fermion der zweiten Generation. Es ist ein geladenes Lepton mit Spin $\frac{1}{2}$  und kann alls schwerere Version des Elektrones bezeichnet werden.
Die Masse des myon beträgt $m_{\mu} = \SI{105}{\MeV}$ und es ist somit etwa $200$ mal schwerer als das Elektron.
Im gegensatz zum Elektron ist es instabil und zerfällt hauptsächlich über den Prozess
\begin{equation*}
    \mu^- \rightarrow e^- \bar \nu_e \nu_{\mu}.
\end{equation*}
An der Erdorberfläche lassen sich kosmische Myonen messen, welche aus Pionen-Zerfällen in der Atmosphäre entstehen.
\subsection{Myon-Zerfall}
Der Myon-Zerfall kann durch das Zerfallsgesetz 
\begin{equation*}
    \mathrm{d} N = -\Gamma \mathrm{d}t
\end{equation*}
beschrieben werden, wobei $N $ die Anzahl der Myonen beschriebt und $\Gamma $ die Zerfallsbreite ist.
Die Differentialgleichung lässt sich durch 
\begin{equation}
    \label{eqn:Zerfall}
    N(t) = N_0 \exp \left(-\Gamma t \right)
\end{equation}
lösen, wo $N_0$ die Anzahl der Teilchen bei $t = 0$ ist. Die Zerfallsbreite steht im Zusammenhang mit der mittleren Lebensdauer $ \tau = \frac{1}{\Gamma}$. Die mittlere Lebensdauer sagt aus, dass nach einer Zeit $\tau$ die Anzahl um einen Faktor $1/e$ gesunken ist.