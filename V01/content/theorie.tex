\section{Zielsetzung}
Ziel des Versuchs ist die Messung der Lebensdauer kosmischer Myonen, die mit einem Szintillationsdetektor
nachgewiesen werden.
\section{Theorie}
Innerhalb des Standardmodells der Teilchenphysik ist das Myon ein Fermion der zweiten Generation. Es ist ein geladenes Lepton mit Spin $\frac{1}{2}$  und kann alls schwerere Version des Elektrons bezeichnet werden.
Die Masse des Myons beträgt $m_{\mu} = \SI{105}{\MeV}$, es ist somit etwa $200$ mal schwerer als das Elektron \cite{pdg}.
Im Gegensatz zum Elektron ist es instabil und zerfällt hauptsächlich über den Prozess
\begin{equation*}
    \mu^- \rightarrow e^- \bar \nu_e \nu_{\mu}.
\end{equation*}
An der Erdoberfläche lassen sich kosmische Myonen messen, welche aus Pion-Zerfällen in der Atmosphäre entstehen.
\subsection{Lebensdauer von Myonen}
Der Myon-Zerfall kann durch das Zerfallsgesetz 
\begin{equation*}
    \mathrm{d} N = -\upGamma \mathrm{d}t
\end{equation*}
beschrieben werden, wobei $N $ die Anzahl der Myonen beschreibt und $\upGamma$ die Zerfallsbreite ist.
Die Differentialgleichung lässt sich durch 
\begin{equation}
    \label{eqn:Zerfall}
    N(t) = N_0 \exp \left(-\upGamma t \right)
\end{equation}
lösen, wobei $N_0$ die Anzahl der Teilchen bei $t = 0$ ist. Die Zerfallsbreite hängt über  $ \tau = \frac{1}{\upGamma}$ mit der mittleren Lebensdauer zusammen. 
Die mittlere Lebensdauer sagt aus, dass nach einer Zeit $\tau$ die Anzahl der Teilchen um den Faktor $1/e$ gesunken ist.