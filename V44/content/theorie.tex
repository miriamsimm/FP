\section{Theorie}
Röntgenstrahlung ist für das menschliche Auge nicht sichtbare Strahlung mit einer Wellenlänge unter 10 nm.
In einer Röntgenröhre entsteht Röntgenstrahlung durch zwei verschiedene Prozesse durch Beschleunigung von 
Elektronen von einer Kathode zur Anode in einem elektrischen Feld. Zum einen schlagen die Elektronen
beim Auftreffen auf die Anode Elektronen aus den inneren Schalen der Atome des Anodenmaterials heraus. 
Durch Übergänge von Elektronen höhrerer Energieniveaus werden Photonen entsprechend der 
freiwerdenen Bindungsenergie emittiert. Es entsteht ein Linienspektrum, welches charakteristisch für das 
Anodenmaterial ist, das charakteristische Röntgenspektrum. 
Zum anderen entsteht Röntgenstrahlung auch bei der Beschleunigung oder Abbremsung von Elektronen. Die 
Wellenlänge der so erzeugten Bremsstrahlung ist dabei abhängig von der Stärke der Beschleunigung bzw. 
Abbremsung und korrespondiert so zu der zwischen Kathode und Anode angelegten Beschleunigungsspannung.
Das Spektrum der Bremsstrahlung ist kontinuierlich und dem charakteristischen Spektrum überlagert.

Röntgenstrahlung hat einen Brechungsindex
\begin{equation*}
    n = 1 - \delta + i \beta
\end{equation*}
mit einem kleinen positiven Störungsparameter $\delta$ mit typischer Größenordnung 
$\delta \sim \mathcal{O}(10^{-6})$. Der Realteil $\text{Re}(n) = 1-\delta$ beschreibt die Brechung, der
Imaginärteil $\text{Im}(n) = \beta$ die Absorption. Durch den Parameter $\delta$ wird der Realteil des 
Brechungsindex kleiner als 1, sodass Totalreflexion möglich ist.