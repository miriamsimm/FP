\section{Durchführung}
In diesem Versuch wird zur Messung der Reflektivität ein D8-Labordiffraktometer verwendet, ein 
$\theta$ - $\theta$ - Diffraktometer, bei dem Röntgenröhre und Detektor beliebig um den Probentisch 
gedreht werden können. Die zur Erzeugung der Röntgenstrahlung verwendete Kupferanodenröhre wird bei einem
Strom von $\SI{35}{\mA}$ und einer Spannung von $\SI{40}{\kV}$ betrieben. Mit einem Göbel-Spiegel 
wird die emittierte divergente Strahlung parallelisiert. %Intensität und Versuchsaufbau?

Die Datennahme erfolgt mit dem Computerprogramm "XRD Commander". 
Vor Beginn der Messung wird die Probe justiert. Dies erfolgt mit einem mehrschrittigen Verfahren 
verschiedener Scans, die einzelnen Justageschritte sind in Tab. \ref{tab:scanverfahren} aufgeführt.
\FloatBarrier
\begin{table}[h]
    \centering
    \caption{Übersicht der einzelnen Schritte bei der Justierung der Messapparatur. Die Messdauer pro Messpunkt beträgt in jedem Schritt 1s.}
    \label{tab:scanverfahren}
    \begin{tabular}{l S[table-format=2.2]@{$-$}S[table-format=1.1] S[table-format=1.3]}
        \toprule
        {Scan-Typ}         & \multicolumn{2}{c}{Messbereich} & {Schrittweite} \\
        \midrule
        {Detektorscan}    & -0.5 & 0.5 & 0.02 \\
        {$z$-Scan}         & -1.0 & 1.0 & 0.04 \\
        {Rockingscan $2\theta = 0$}     & -1.0 & 1.0 & 0.04 \\
        {$z$-Scan}         & -0.5 & 0.5 & 0.02 \\
        {Rockingscan $2\theta = 0,3$}     & 0    & 3.0 & 0.005 \\
        {$z$-Scan $2\theta = 0,3$}         & -0.5 & 0.5 & 0.02 \\
        \bottomrule
    \end{tabular}
\end{table}
\FloatBarrier
\noindent
Zuerst wird mit einem Detektorscan der Primärstrahl justiert. Vorab wird die Probe in $z$-Richtung aus 
dem Strahl gefahren und die Röhre auf 0° bewegt. Die Nulllage des Detektors wird bestimmt, indem der
Detektor in einem kleinen Winkelbereich um die Position der Röhre variiert wird. Der Scan hat die Form 
einer Gaußglocke, dessen Maximum der Nullposition entspricht. Anschließend wird durch Verschieben des Probentisches
in alle drei Raumrichtungen die Probe nach Augenmaß positioniert. Die gesuchte Position ist erreicht, wenn 
die Probe parallel zum Strahl steht und die halbe Intensität des Primärstrahls abschattet. Dazu wird ein
$z$-Scan durchgeführt, der die $z$-Position der Probe ändert und mit dem so durch eine Intensitätsmessung 
die halbe Abschattung bei $I = \frac{1}{2} I_\text{max}$ bestimmt wird. Der anschließende Rockingscan
dient zur Justierung der Verkippung der Probe zum Strahl. Dabei werden die Röntgenröhre und der Detektor 
so um die Probe gedreht, dass der Winkel zwischen Röhre und Detektor konstant bleibt. Die Position des 
Probentisches wird so lange variiert, bis die gemessene Intensitätsverteilung einem symmetrischen Dreieck 
entspricht. Mit einem weiteren $z$-Scan, Rockingscan und anschließendem $z$-Scan wird eine noch präzisere 
Justierung der Probe erreicht. 

Mit einem Reflektivitätsscan erfolgt nun die Untersuchung der Polysterolschicht auf dem Silizium-Wafer. 
Bei der Messung sind der Einfallswinkel auf die Probe und der Winkel zwischen Probe und Detektor gleich.
Der Scan erfolgt im Bereich von 0° bis 2,5° bei einer Schrittweite von 0,005° und einer Messzeit von 
5s pro Messpunkt. Mit einem Diffuserscan wird der Anteil der gestreuten Intensität an der Reflektivität
bestimmt, indem in einer zweiten, analog verlaufenden Messung der Detektorwinkel um 0,1° gegenüber dem 
Einfallswinkel verschoben wird.