\section{Durchführung}
In diesem Versuch wird zur Messung der Reflektivität ein D8-Labordiffraktometer verwendet, ein 
$\theta$ - $\theta$-Diffraktometer bei dem Röntgenröhre und Detektor beliebig um den Probentisch 
gedreht werden können. Die zur Erzeugung der Röntgenstrahlung verwendete Kupferanodenröhre wird bei einem
Strom von $\SI{35}{\mA}$ und einer Spannung von $\SI{40}{\kV}$ betrieben. Mit einem Göbel-Spiegel 
wird die emittierte divergente Strahlung parallelisiert. %Intnsität und Versuchsaufbau?

Die Datennahme erfolgt mit dem Computerprogramm "XRD Commander". 
Vor Beginn der Messung wird die Probe justiert. Dies erfolgt mit einem mehrschrittigen Verfahren 
verschiedener Scans, siehe Tabelle \ref{tab:scanverfahren}